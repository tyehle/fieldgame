\documentclass[twoside,12pt]{article}
\usepackage{moreverb}
\usepackage{amsmath}
\usepackage{tikz}
\usepackage{tikz-3dplot}
\usepackage{asymptote}
\usepackage{graphicx}
\usepackage{amsfonts}
\usepackage{appendix}
\usepackage[margin=1in]{geometry}
\begin{document}
\title{Projection}
\author{Tobin Yehle}
\date{August 2019}
\maketitle

\newcommand{\dotp}{\boldsymbol{\cdot}}

\begin{center}
\begin{asy}[width=0.75\textwidth]
    settings.render=0;
    //settings.outformat="png";
    settings.prc=false;
    import three;
    import graph3;
    import grid3;
    currentprojection=orthographic(-1,-2,-4,up=-Y);
    
    
    // Styles
    pen thickblack = black+0.75;
    pen thinblack = black+0.25;
    pen thingray = gray+0.33;
    
    // Draw Axes
    
    // a nearly transparent sphere to help see perspective
    triple rin(pair t){ return expi(t.x,t.y); }
    surface sphere=surface(rin,(-pi/2,-pi/2),(pi/2,pi/2),32,32);
    draw(sphere, emissive(gray+opacity(0.125)));
    
    // R
    draw(L=Label("$\mathbf{r}$", position=Relative(1.1), align=E), O--X, thickblack, Arrow3);
    draw(O--(-1,0,0), p=thingray, arrow=None);
    
    // D
    draw(L=Label("$\mathbf{d}$", position=Relative(1.1), align=S), O--Y, thickblack, Arrow3);
    draw(O--(0,-1,0), p=thingray, arrow=None);
    
    // F
    draw(L=Label("$\mathbf{f}$", position=Relative(1.1), align=SW), O--Z, thickblack, Arrow3);
    
    
    // Draw latitude lines
    path3 lat(real l) { return circle(Z*cos(l),sin(l),Z); }
    //draw( lat(pi/6), thinblack );
    //draw( lat(pi/3), thinblack );
    //draw( lat(pi/4), thinblack );
    draw( lat(pi/2), thinblack );
    
    // Draw longitude lines
    path3 lon(real l) { return arc(O, 1, 0, 0, 90, l); }
    for(int l=0; l < 360; l+=90) {
      draw( lon(l), thinblack );
    }
    
    
    // p vector
    triple P=dir((5,-3,5));
    draw( L=Label("$\mathbf{\hat{p}}$", position=Relative(1.1), align=N), O--P, thickblack, Arrow3 );
    
    // alpha
    draw( arc(O, Z, P), black );
    draw( L=Label("$\alpha$", align=N), arc(O, 0.2*Z, 0.2*P), black );
    
    // theta
    triple P_f=(0,0,P.z);
    triple lat_r=(sqrt(1-P.z*P.z),0,P.z);
    real theta=atan(P.y/P.x);
    draw( P--P_f--lat_r, dashed );
    draw( arc(P_f, P, lat_r), black );
    draw( L=Label("$\theta$", align=E), arc(P_f, 0.2, 90, theta*180/pi, 90, 0), black );
\end{asy}

\begin{asy}[width=0.75\textwidth]
    settings.render=0;
    //settings.outformat="png";
    unitsize(6cm);
    
    // Styles
    pen thickblack = black+0.75;
    pen thinblack = black+0.25;
    pen thingray = gray+0.33;
    
    draw(unitcircle, gray);
    fill(unitcircle, palegray);
    draw((-1,0) -- (1,0), p=thingray);
    draw((0,-1) -- (0,1), gray);
    
    draw(L=Label("$\mathbf{r}_2$", position=EndPoint), (0,0) -- (1,0), p=thickblack, arrow=Arrow);
    draw(L=Label("$\mathbf{d}_2$", position=EndPoint), (0,0) -- (0,-1), p=thickblack, arrow=Arrow);
    label("$\mathbf{f}_2$", (0,0), align=SW);
    
    // p = (5,-3,5)
    // alpha = 49.39 = 0.8620 rad
    // theta = 30.96 = 0.5404 rad
    // x = 0.4706
    // y = 0.2823
    draw(L=Label("$\mathbf{p}_2$", position=EndPoint), (0,0) -- (0.4706,0.2823), p=thickblack, arrow=Arrow);
    draw(L=Label("$\alpha$", position=MidPoint, align=NW), shift(-0.02823,0.04706) * ((0,0) -- (0.4706,0.2823)), p=thinblack, bar=Bars);
    draw(L=Label("$\theta$", position=MidPoint), arc((0,0), 0.25, 0, 30.96), p=thinblack);
\end{asy}

\begin{asy}[width=0.75\textwidth]
    settings.render=0;
    //settings.outformat="png";
    settings.prc=false;
    import three;
    import graph3;
    import grid3;
    
    //currentprojection=orthographic(-1.5,-2.5,-4,up=-Y);
    currentprojection=orthographic(-1.5,-2,-4,up=-Y);
    
    // Styles
    pen thickblack = black+0.75;
    pen thinblack = black+0.25;
    pen thingray = gray+0.33;
    
    // Draw Axes
    // R
    draw(L=Label("$\mathbf{r}$", position=Relative(1.1), align=SE), O--X, thickblack, Arrow3);
    draw((-1.5,0,0)--(3.25,0,0), p=thingray, arrow=None);
    
    // D
    draw(L=Label("$\mathbf{d}$", position=Relative(1.1), align=SW), O--Y, thickblack, Arrow3);
    draw((0,-2,0)--(0,1.5,0), p=thingray, arrow=None);
    
    // F
    draw(L=Label("$\mathbf{f}$", position=Relative(1.1), align=SW), O--Z, thickblack, Arrow3);
    draw(O--(0,0,3.25), p=thingray, arrow=None);
    
    
    // Draw latitude lines
    path3 lat(real l) { return circle(Z*cos(l),sin(l),Z); }
    //draw( lat(pi/6), thinblack );
    //draw( lat(pi/3), thinblack );
    //draw( lat(pi/4), thinblack );
    draw( lat(pi/2), thinblack );
    
    // Draw longitude lines
    path3 lon(real l) { return arc(O, 1, 0, 0, 90, l); }
    for(int l=0; l < 360; l+=180) {
      draw( lon(l), thinblack );
    }
    //draw( arc(O, 2, -10, 0, 100, 0), thinblack );
    //draw( arc(O, 3, -10, 0, 100, 0), thinblack );
    
    // a nearly transparent sphere to help see perspective
    triple rin(pair t){ return expi(t.x,t.y); }
    surface sphere=surface(rin,(-pi/2,-pi/2),(pi/2,pi/2),32,32);
    draw(sphere,emissive(gray+opacity(0.125)));
    
    
    // p vector
    triple P=(2.5,-1.5,2.5);
    triple P_hat=dir(P);
    triple P_rf=(P.x,0,P.z);
    triple P_r=(P.x,0,0);
    triple P_f=(0,0,P.z);
    draw( O--P_hat, black+1.25 );
    draw( L=Label("$\mathbf{p}$", position=Relative(1.1), align=N), O--P, thickblack, Arrow3 );
    
    //draw( P--P_rf--O, thinblack );
    // or
    draw( P--P_rf, thinblack );
    draw( P_r--P_rf--P_f, thinblack );
    
    //draw( P_rf--P_r, dashed );
    
    // alpha
    draw( L=Label("$\alpha$", align=.5N + 2.5W), arc(O, Z, P_hat), thickblack );
    
    // P reject f
    triple P_rd=(P.x,P.y,0);
    triple P_rd_hat=dir(P_rd);
    //draw( O--P_rd_hat, black+1.25 );
    draw( L=Label("$\mathbf{P}_{/\mathbf{f}}$", position=Relative(1.1), align=E), O--P_rd, thickblack, Arrow3 );
    draw( P_rd--P_r, thinblack );
    draw( P--P_rd, dashed );
    //draw( P_hat--(P_hat.x,P_hat.y,0)--(P_hat.x,0,0), dashed );
    
    // theta
    draw( L=Label("$\theta$", align=E), arc(O, X, P_rd_hat), thickblack );
    triple P_hat_f=(0,0,P_hat.z);
    triple lat_r=(sqrt(1-P_hat.z*P_hat.z),0,P_hat.z);
    //draw( P_hat--P_hat_f--lat_r, dashed );
    //draw( arc(P_hat_f, P_hat, lat_r), black );
    
    // P dot r
    draw( L=Label("$\mathbf{p}\cdot\mathbf{r}$", position=Relative(1.1), align=SE), O--P_r, thickblack, Arrow3 );
    
    // P dot d
    triple P_d=(0,P.y,0);
    draw( L=Label("$\mathbf{p}\cdot\mathbf{d}$", position=Relative(1.1), align=NW), O--P_d, thickblack, Arrow3 );
    draw( P_rd--P_d, dashed );
\end{asy}
\end{center}


The easiest component of the projection to find is the polar coordinate radius, or latitude, $\alpha$.

\[
\alpha = \arccos \left( \frac{\mathbf{p}}{\|\mathbf{p}\|}  \dotp \mathbf{f} \right)
\]

We will use the rejection of the forward vector $\mathbf{f}$ from $\mathbf{p}$ to find the equatorial rotation of the projected point. This rejection vector is equivalent to the projection of $\mathbf{p}$ onto the plane orthogonal to $\mathbf{f}$.

\[
\mathbf{p}_{/\mathbf{f}} = \mathbf{p} - \mathbf{f} ( \mathbf{p} \dotp \mathbf{f} )
\]

The angle, $\theta$, between the right vector, $\mathbf{r}$, and $\mathbf{p}_{/\mathbf{f}}$ is the angle component of the of the polar coordinate projection.


The cartesian coordinates in $\mathbb{R}^2$ can then be expressed in terms of $\alpha$ and $\theta$

\[ x = \alpha \cos ( \theta ) \]
\[ y = \alpha \sin ( \theta ) \]

The trig operations on $\theta$ can be reduced because we have access to the lengths of the sides of a triangle containing $\theta$.

\[ \cos ( \theta ) = \frac{ \mathbf{p} \dotp \mathbf{r} }{ \| \mathbf{p}_{/\mathbf{f}} \| } \]
\[ \sin ( \theta ) = \frac{ \mathbf{p} \dotp \mathbf{d} }{ \| \mathbf{p}_{/\mathbf{f}} \| } \]

The down vector, $\mathbf{d}$, is not stored, but can be found with a cross product.

\[ \mathbf{d} = \mathbf{f} \times \mathbf{r} \]

Putting all these pieces together gives the cartesian coodinates of the projected point.

\[ x = \alpha \; \frac{ \mathbf{p} \dotp \mathbf{r} }{ \| \mathbf{p}_{/\mathbf{f}} \| } \]

\[ y = \alpha \; \frac{ \mathbf{p} \dotp ( \mathbf{f} \times \mathbf{r} ) }{ \| \mathbf{p}_{/\mathbf{f}} \| } \]

\end{document}